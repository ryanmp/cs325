\documentclass[letterpaper,10pt,titlepage]{article}

\usepackage{graphicx}                                        

\usepackage{amssymb}                                         
\usepackage{amsmath}                                         
\usepackage{amsthm}                                          

\usepackage{alltt}                                           
\usepackage{float}
\usepackage{color}

\usepackage{url}

\usepackage{balance}
\usepackage[TABBOTCAP, tight]{subfigure}
\usepackage{enumitem}

\usepackage{pstricks, pst-node}

\usepackage{geometry}
\geometry{textheight=9in, textwidth=6.5in}

%random comment

\newcommand{\cred}[1]{{\color{red}#1}}
\newcommand{\cblue}[1]{{\color{blue}#1}}

\usepackage{hyperref}

\def\name{Joshua Villwock}

%pull in the necessary preamble matter for pygments output
\input{pygments.tex}

%% The following metadata will show up in the PDF properties
\hypersetup{
  colorlinks = true,
  urlcolor = black,
  pdfauthor = {\name},
  pdfkeywords = {cs311 ``operating systems'' files filesystem I/O},
  pdftitle = {CS 311 Project 6: Socket I/O},
  pdfsubject = {CS 311 Project 6},
  pdfpagemode = UseNone
}

\parindent = 0.0 in
\parskip = 0.2 in

\begin{document}
\tableofcontents

\section{Overall System Design}

There are 8 Used packet types in the system:

\begin{center}
    \begin{tabular}{ | l | l | l | p{5cm} |}
    \hline
    Packet ID & <-> & Name & Description \\ \hline
    0 & c <-> s & Keep-Alive & Apacket That is Completly Unused but   
    The server will ping it back to the client if received. \\ \hline
    1 & c -> s & Range Request & The Client sends this to the server
    To ask for work to do.  Payload is the range
    of numbers the client would like \\ \hline
    2 & c <- s & Assign Range & Server sends this back to the client
    Payload is the number the client should start at
    It is implied that the Client should do how much they requested past that \\ \hline
	3 & c -> s & Found Number & Client Sends this to server When Number Found \\ \hline
	4 & c -> s & Unused & Unused \\ \hline
	5 & c <-> s & Report Found & Reporter Uses to ask Server What numbers were found \\ \hline
	6 & c <-> s & Report Clients & Reporter Uses to ask Server What Clients are connected \\ \hline
	7 & <-> & Report Number & Reporter Uses to ask Server What Number we are currently on \\ \hline
	8 & - & Unused & Unused \\ \hline
	9 & c <-> s & Kill & A server can send this to the clients to tell them to shut down
    or a reporter can send this to the Server to tell it to stop
	Its clients and shut down.\\ \hline
    \end{tabular}
\end{center}

Note that the Client is not written in C, due to time contraints and problems.  You can read below for more details.

However, some source code I was working on for a C based client is included, but definitly not working.  Check README.md for how to use it, or see below for more details.

\section{Work Log}

Nov 23- Python Socket code

Nov 26- Server keeps listening

Nov 28- Server async

Dec 1- Finished Basic Client server Working

Dec 2- Fixed Stuff, Wrote Report

Full work log can be found here:
\url{https://github.com/1n5aN1aC/Perfect/commits/master}

\section{Challenges}

I tried writing the client in C, as specified, but I both had issues with witing a proper asncronous network client for it, and getting json parson and construction working in it.

The pythn was fairly easy, except for a few quirks of python that had to be learned, and overcome.

\section{Questions And Answers}

\subsection{what do you think the main point of this assignment is?}

\begin{itemize}
\item Mainly, to get us to learn to develop our own protocol, and implement it.  This implies a lot of engeneering, and planning
\item But also, to use threads, or async network programming, to amke sure the IO is isolated fromt he Program logic
\end{itemize}

\subsection{how did you ensure your solution was correct? Testing details, for instance.}

Tests were ran on many different computers, with the server only being tested on my computer, but tested locally, over the LAN, and over the internet.

It was tested with as many as 8 consecutive clients.

\subsection{what did you learn?}

I learned Async Network programming in python is fairly fun, but much, much, harder in C or C Plus Plus.

\end{document}
